% コンパイル方法: lualatex filename.tex
\documentclass[a4paper,11pt]{article}

% マージン設定
\usepackage[top=20mm,bottom=25mm,left=20mm,right=20mm]{geometry}

% LuaLaTeX用日本語対応パッケージ
\usepackage{luatexja}
\usepackage{luatexja-fontspec}

% 必要なパッケージ
\usepackage{fontspec}
\usepackage{titlesec}
\usepackage{graphicx}
\usepackage{amsmath}
\usepackage{hyperref}
\usepackage[english,japanese]{babel}
\usepackage{multicol} % 二段組用パッケージ
\usepackage{indentfirst}
\usepackage{tikz} % カスタム点線用


% フォント設定
\newjfontfamily\TitleJFont{BIZ UDPGothic}[BoldFont=BIZ UDPGothicBold]
\newjfontfamily\SectionJFont{BIZ UDPGothic}[BoldFont=BIZ UDPGothicBold]


% セクション見出しのカスタマイズ
\titleformat{\section}
  {\large\bfseries\SectionJFont}
  {\thesection}
  {1em}{}


% customabstract 環境を定義
\newenvironment{customabstract}
{\noindent\hfuzz=10pt\hbadness=10000\begin{flushleft}\small}
{\end{flushleft}}


% タイトル情報
\title{{\TitleJFont\bfseries 感情分析と時系列モデルを統合した\\データ駆動型予測システムの開発 \\ }}
\author{\large\TitleJFont\bfseries 東京都立産業技術高等専門学校 ものづくり工学科 情報通信工学コース \\ \large\TitleJFont\bfseries 下沢 亮太郎}
\date{}

\titlespacing*{\section}{0pt}{8pt}{5pt}

\begin{document}

% タイトルの表示
% \maketitle
% \thispagestyle{empty}

% % アブストラクト部分

\setlength{\columnsep}{40pt}

\twocolumn[
  \maketitle

  \vspace{-2em}
  \begin{customabstract}
   キーワード : 機械学習 時系列予測 感情分析 BERT LSTM
  \end{customabstract}
]

% 二段組の開始
% \begin{multicols*}{2}


\section{緒言}
社会的な事象に影響を与えることが多いのは文字列や映像なことが多いという事実に対して有名な時系列予測モデルは入力に数値しかとらないことが多い.

そこで,本研究ではニュースの文章をBERTという自然言語に特化した分類モデルが出力した値を「感情らしさ」として採用しよく使われる指標とともに時系列予測に特化したLSTMという機械学習モデルで学習し,予測した.

\section{BERTの転移学習}
BERT(Bidirectional Encoder Representations from Transformers)はGoogleによって事前学習されたモデルをクライアント側で用意したデータセットで転移学習(Fine-tuning)することで任意の形式で出力するモデルを作ることができる.



\section{図の例}
図を以下のように挿入できます。
\begin{figure}[h]
    \centering
    \includegraphics[width=1\columnwidth]{example-image} % 適切な画像パスを指定
    \caption{サンプル画像}
\end{figure}

\section{結言}
ご質問がありましたら以下にご連絡ください。

% カスタム点線を描画
\noindent
\begin{tikzpicture}
\draw[dotted, thick] (0,0) -- (\linewidth,0);
\end{tikzpicture}
お問い合わせ先\\
氏名:下沢 亮太郎 \\
E-mail : \href{mailto:example@example.ac.jp}{example@example.ac.jp}

% \end{multicols*}

\end{document}
