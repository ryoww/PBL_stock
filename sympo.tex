% コンパイル方法: lualatex filename.tex
\documentclass[a4paper,11pt]{article}

% マージン設定
\usepackage[top=20mm,bottom=25mm,left=20mm,right=20mm]{geometry}

% LuaLaTeX用日本語対応パッケージ
\usepackage{luatexja}
\usepackage{luatexja-fontspec}

% 必要なパッケージ
\usepackage{fontspec}
\usepackage{titlesec}
\usepackage{graphicx}
\usepackage{amsmath}
\usepackage{hyperref}
\usepackage[english,japanese]{babel}
\usepackage{multicol} % 二段組用パッケージ
\usepackage{indentfirst}

% フォント設定
\newjfontfamily\TitleJFont{BIZ UDPGothic}[BoldFont=BIZ UDPGothicBold]
\newjfontfamily\SectionJFont{BIZ UDPGothic}[BoldFont=BIZ UDPGothicBold]

% セクション見出しのカスタマイズ
\titleformat{\section}
  {\large\bfseries\SectionJFont}
  {\thesection}
  {1em}{}

% customabstract 環境を定義
\newenvironment{customabstract}
{\noindent\hfuzz=10pt\hbadness=10000\begin{flushleft}\small}
{\end{flushleft}}

% タイトル情報
\title{{\TitleJFont\bfseries 感情分析と時系列モデルを統合したデータ駆動型予測システム \\ }}
\author{\large\TitleJFont\bfseries 東京都立産業技術高等専門学校 ものづくり工学科 情報通信工学コース \\ \large\TitleJFont\bfseries 下沢 亮太郎}
\date{}

\begin{document}

% タイトルの表示
\maketitle
\thispagestyle{empty}

% アブストラクト部分
\begin{customabstract}
 キーワード : 機械学習 時系列予測 感情分析 BERT LSTM
\end{customabstract}

\setlength{\columnsep}{40pt}

% 二段組の開始
\begin{multicols}{2}

\section{緒言}
このテンプレートは、簡潔な日本語文書を作成するためのものです。アブストラクト部分は一段組で、それ以降は二段組の構成を採用しています。

\section{数式の例}
以下に数式の例を示します。
\begin{equation}
E = mc^2
\end{equation}

\section{図の例}
図を以下のように挿入できます。
\begin{figure}[h]
    \centering
    \includegraphics[width=0.8\columnwidth]{example-image} % 適切な画像パスを指定
    \caption{サンプル画像}
\end{figure}

\section{結言}
ご質問がありましたら以下にご連絡ください。

\noindent\hrulefill\\
お問い合わせ先\\
氏名:下沢 亮太郎 \\
E-mail : \href{mailto:example@example.ac.jp}{example@example.ac.jp}

\end{multicols}

\end{document}
